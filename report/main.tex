% ======= Réglages =======
% config.tex
% Contient toutes les commandes de configurations 
% Pour la template LATEX

% Type du document
\documentclass[11pt]{article}

% ---- Bibliographie ---------
% \usepackage{natbib} 

% ------- Français -----------	
\usepackage[francais]{babel}
\usepackage[utf8]{inputenc}
\usepackage[T1]{fontenc}

\usepackage{titling}

\usepackage{graphicx}
\usepackage{subfigure}
\usepackage{float}
\usepackage{lmodern}
\usepackage{ae,aecompl}
\usepackage{enumitem}
\usepackage{scalerel}
\usepackage{textcomp}

\usepackage{listings}


\lstdefinestyle{customPseudo}{belowcaptionskip=1\baselineskip,
                              breaklines=true,
                              frame=simple,
                              xleftmargin=\parindent,
                              showstringspaces=false,
                              basicstyle=\footnotesize\ttfamily,
                              keywordstyle=\bfseries\color{blue!40!black},
                              commentstyle=\itshape\color{green!40!black},
                              identifierstyle=\color{black},
                              stringstyle=\color{OrangeRed}
                            }

\lstdefinestyle{customMatlab}{belowcaptionskip=1\baselineskip,
                              breaklines=true,
                              frame=simple,
                              xleftmargin=\parindent,
                              language=Matlab,
                              showstringspaces=false,
                              basicstyle=\footnotesize\ttfamily,
                              keywordstyle=\bfseries\color{blue!40!black},
                              commentstyle=\itshape\color{green!40!black},
                              identifierstyle=\color{black},
                              stringstyle=\color{OrangeRed}}

\lstdefinestyle{customPython}{belowcaptionskip=1\baselineskip,
                              breaklines=true,
                              frame=simple,
                              xleftmargin=\parindent,
                              language=Python,
                              showstringspaces=false,
                              basicstyle=\footnotesize\ttfamily,
                              keywordstyle=\bfseries\color{blue!40!black},
                              commentstyle=\itshape\color{green!40!black},
                              identifierstyle=\color{black},
                              stringstyle=\color{OrangeRed}}

% ======================
%	     COULEURS
% Couleur sur les titres
\usepackage[dvipsnames]{xcolor}
\usepackage{sectsty}

\sectionfont{\color{Red}}
\subsectionfont{\color{Maroon}}
\subsubsectionfont{\color{Periwinkle}}  % sets colour of subsections

% Couleur pour le rectangle du titre
\definecolor{grey}{rgb}{0.9,0.9,0.9} % Color of the box surrounding the title
% ---- Hyperliens --------
\usepackage{hyperref}
\hypersetup{
     colorlinks   = true,
     citecolor    = blue
}
\hypersetup{linkcolor=NavyBlue}




% =========================
%    	MISE EN PAGE
	\usepackage{fancyhdr}
	\pagestyle{fancy}
% Clears the default layout
	\fancyhead{}
	\fancyfoot{}
% Own layout
	\rhead{}

	\fancyfoot[L] {\thepage}
	\renewcommand{\headrulewidth}{0pt}
	\renewcommand{\footrulewidth}{0.4pt}




% --------------------------------	
%	Caractères mathématiques
\usepackage{amsmath}
\usepackage{systeme}
\usepackage{amsfonts}
\usepackage{mathabx}

% Vector column type
\newcount\colveccount
\newcommand*\colvec[1]{
        \global\colveccount#1
        \begin{pmatrix}
        \colvecnext
}
\def\colvecnext#1{
        #1
        \global\advance\colveccount-1
        \ifnum\colveccount>0
                \\
                \expandafter\colvecnext
        \else
                \end{pmatrix}
        \fi
}
   
% --------------------------------	
% customisation des puces de liste
\renewcommand{\labelitemi}{$\cdot$}
\renewcommand{\labelitemii}{$\star$}



% ======================
% 		GEOMETRY
\usepackage[a4paper,pdftex]{geometry}                		% See geometry.pdf to learn the layout options. There are lots.
\geometry{a4paper} 
% \geometry{landscape}	% Activate for rotated page geometry



% ======================
% 		GRAPHIC
\usepackage{graphicx} % Use pdf, png, jpg, or eps§ with pdflatex; use eps in DVI mode
							% TeX will automatically convert eps --> pdf in pdflatex		
\usepackage{amssymb}
\usepackage{xcolor}
%SetFonts



% ======================
% 		MARGINS
\setlength{\oddsidemargin}{0mm} % Adjust margins to center the colored title box
\setlength{\evensidemargin}{0mm} % Margins on even pages - only necessary if adding more content to this template

\newcommand{\HRule}[1]{\hfill \rule{0.2\linewidth}{#1}} % Horizontal rule at the bottom of the page, adjust width here




\title{Fondamentaux pour ATIAM: informatique}
\author{Victor WETZEL}
\date{\today{month}}						

% ========================
\begin{document}
\title{}

\lstset{
        inputencoding=latin1,
        basicstyle=\footnotesize\ttfamily,
        showstringspaces=false,
        showspaces=false,
        numbers=left,
        numberstyle=\footnotesize,
        numbersep=9pt,
        tabsize=2,
        breaklines=true,
        showtabs=false,
        captionpos=b,
        literate={é}{{\'e}}1 {è}{{\`e}}1 {ê}{{\^e}}1
       }
\pagenumbering{gobble}
\colorbox{Red}{
	\parbox[t]{1.0\linewidth}{
		\centering \fontsize{35pt}{80pt}\selectfont % The first argument for fontsize is the font size of the text and the second is the line spacing - you may need to play with these for your particular title
		\vspace*{0.7cm} % Space between the start of the title and the top of the grey box
		
        
        \hfill \LARGE{\textbf{FpA: Informatique}} \\
 
 		\vspace{10pt}
		\hfill \Large{Victor Wetzel}\par
		\hfill \Large{\textit{Fondamentaux pour ATIAM}}  
		\vspace*{0.7cm} % Space between the end of the title and the bottom of the grey box
	}
}

%----------------------------------------------------------------------------------------
\vspace{70pt}

\vfill % Space between the title box and author information

%----------------------------------------------------------------------------------------
%	AUTHOR NAME AND INFORMATION SECTION
%----------------------------------------------------------------------------------------

{\centering \large 
\hfill Master 2 SpI, ATIAM\\
\hfill Septembre-Octobre 2017\\
}

%----------------------------------------------------------------------------------------

\clearpage % Whitespace to the end of the page

% Remove page numbers (and reset to 1)
\newpage
\tableofcontents

\newpage
\pagenumbering{arabic}% Arabic page numbers (and reset to 1)

% ========================
\section{Implémentation de l'algorithme d'alignement de Neddleman-Wunsch}
L'objet de cette partie est de donner quelques éléments sur mon implémentation
de l'algorithme de Needle-Wunsch.\\
On rappelle que cet algorithme permet de trouver les meilleurs alignements de
chaînes de charactères symboliques (?). L'algorithme s'appuie sur la
construction d'une matrice de similarité entre les deux éléments que l'on veut
aligner. J'ai opté pour une approche orientée objet. Un objet \textit{matrix}
est créé pour chaque alignement. Cet objet contient les méthodes permettant de
résoudre l'algorithme d'alignement et d'obtenir une liste de tous les
alignements possibles trié par ordre décroissant de score: le meilleur est en
tête de liste.

\subsection{Objet: \textit{nw.slot}}
Chaque emplacement de la matrice contient un objet \textit{slot}. Ses variables
de champs sont:
\begin{itemize}
  \item{Un couple de symboles} il s'agit des deux caractères que l'on aligne à
    cet endroit particulier de la matrice.
  \item{Une pénalité} Score à retrancher aux cases adjacentes en fonction d'une
    correspondance ou non-correspondance des deux
    caractères (\textit{match} ou \textit{mismatch})
  \item{Un score} Calculé en fonction des case: en haut, en haut à gauche,
    à gauche. à chacune des cases adjacentes listées est ajoutée une
    pénalité en fonction de. Le score de la case est le
    maximum de ces trois sous-scores.
  \item{Une direction}  L'algorithme mémorise l'origine de ce score. 
\end{itemize}

\subsection{Objet: \textit{nw.matrix}}
La matrice elle même est un objet. Les variables de champs sont:
\begin{itemize}
  \item{matrix} la matrice de similarité contenant les \textit{slot}
  \item{N et M} caractérisent la taille de la matrice
  \item{path} Liste tous les alignements possibles
  \item{strA strB} Les chaînes de symboles que l'on compare
\end{itemize}

% =============
% ==== BDD ====
\newpage
\section{Opérations sur la base de donnée}
\subsection{Tri de la base de donnée par ordre décroissant du nombre de tracks}
Dans cette partie, j'ai implémentée l'algorithme de \textit{quicksort} de
Lomuto, en inversant l'ordre du tri. Le pseudo code de l'algorithme original 
est disponible sur la page
wikipédia (https://en.wikipedia.org/wiki/Quicksort), listé ci-dessous:

\hspace{10pt}
\begin{lstlisting}[style=customPseudo, caption=Algorithme quicksort de Lomuto]
algorithm quicksort(A, lo, hi) is
    if lo < hi then
        p := partition(A, lo, hi)
        quicksort(A, lo, p-1)
        quicksort(A, p + 1, hi)

algorithm partition(A, lo, hi) is
    pivot := A[hi]
    i := lo - 1    
    for j := lo to hi - 1 do
        if A[j] < pivot then
            i := i + 1
            swap A[i] with A[j]
    if A[hi] < A[i + 1] then
        swap A[i + 1] with A[hi]
    return i + 1
\end{lstlisting}

Pour trier la base de donnée, on créé une matrice dont la première colonne contient le
nombre de track de chaque artiste, la deuxième contenant le nom de l'artiste.
Il suffit alors d'appliquer la fonction \textit{revQuicksort} sur cette liste
afin de la trier dans l'ordre décroissant du nombre de tracks.

% ========================
\newpage
\bibliographystyle{plain}
\bibliography{bibliographie.bib}


\end{document}
